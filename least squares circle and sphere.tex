\documentclass[11pt,letterpaper,onecolumn]{IEEEtran}

\usepackage{cite}			% cites the references at the end of document
\usepackage{graphicx}
\usepackage{float}    		% tables and other floats
\usepackage{caption}
\usepackage{subcaption}
\usepackage{verbatim}		% comment out large sections similar to /* */
\usepackage{url}			% for hyperlink to other locations within pdf
\usepackage{tikz}			% diagrams
\usepackage{amsmath}
\usepackage{amssymb}
\usepackage{cancel}			% cross-out terms in math mode
\usepackage{listings} 		% For source code

% define hyperref settings
\usepackage{hyperref}
\hypersetup{
	bookmarks=true,         % show bookmarks bar?
	unicode=false,          % non-Latin characters in Acrobat’s bookmarks
	pdftoolbar=true,        % show Acrobat’s toolbar?
	pdfmenubar=true,        % show Acrobat’s menu?
	pdffitwindow=false,     % window fit to page when opened
	pdfstartview={FitH},    % fits the width of the page to the window
	pdftitle={My title},    % title
	pdfauthor={Author},     % author
	pdfsubject={Subject},   % subject of the document
	pdfcreator={Creator},   % creator of the document
	pdfproducer={Producer}, % producer of the document
	pdfkeywords={keyword1, key2, key3}, % list of keywords
	pdfnewwindow=true,      % links in new PDF window
	colorlinks=false,       % false: boxed links; true: colored links
	linkcolor=red,          % color of internal links (change box color with linkbordercolor)
	citecolor=green,        % color of links to bibliography
	filecolor=magenta,      % color of file links
	urlcolor=cyan           % color of external links
}

% define source code settings
\definecolor{mygrey}{gray}{.96} % Light Grey
\lstset{
	language=[ISO]C++,              % choose the language of the code ("language=Verilog" is popular as well)
	tabsize=3,							  % sets the size of the tabs in spaces (1 Tab is replaced with 3 spaces)
	basicstyle=\tiny,               % the size of the fonts that are used for the code
	numbers=left,                   % where to put the line-numbers
	numberstyle=\tiny,              % the size of the fonts that are used for the line-numbers
	stepnumber=2,                   % the step between two line-numbers. If it's 1 each line will be numbered
	numbersep=5pt,                  % how far the line-numbers are from the code
	backgroundcolor=\color{mygrey}, % choose the background color. You must add \usepackage{color}
	%	%showspaces=false,              % show spaces adding particular underscores
	%	%showstringspaces=false,        % underline spaces within strings
	%	%showtabs=false,                % show tabs within strings adding particular underscores
	frame=single,	                 % adds a frame around the code
	tabsize=3,	                    % sets default tabsize to 2 spaces
	captionpos=b,                   % sets the caption-position to bottom
	breaklines=true,                % sets automatic line breaking
	breakatwhitespace=false,        % sets if automatic breaks should only happen at whitespace
	%	%escapeinside={\%*}{*)},        % if you want to add a comment within your code
	commentstyle=\color{blue}   % sets the comment style
}

% define tikzstyle shortcuts
\usetikzlibrary{positioning,shapes,arrows}
\tikzstyle{Algo} = [rectangle, rounded corners, minimum width=1 cm, minimum height=1cm, text centered, draw=black]
\tikzstyle{Process} = [rectangle, minimum width=1 cm, minimum height=1cm, text centered, draw=black, fill=blue!20]
\tikzstyle{Sum} = [circle, radius = 3 cm, text centered, draw=black]
\tikzstyle{Arrow} = [thick,->,>=stealth]
\tikzstyle{Line} = [thick]
\tikzstyle{Junction} = [coordinate]

\begin{document}
\title{Least-squares Fit to Circles and Spheres}
\author{Lisandro Leon}
\maketitle

%\abstract
%This is the abstract.

\begin{comment}
\section{Problem Statement}
\begin{figure}[!h]
	\center{
		\includegraphics[width=0.2\textwidth]
		{three-link planar serial manipulator.png}
	}
	\caption{loop-shape superimposed as a visual tool for controller design}\label{fig2}
\end{figure}
\end{comment}

\section{Least-squares formula}
If a set of simultaneous, linear equations can be formulated as
\begin{equation}
\mathbb{A} \mathbf{x} = \mathbf{y},\label{eq0}
\end{equation}
and if all the columns of $\mathbb{A}$ are linearly independent, then there exists only one least-squares solution $\hat{\mathbf{x}}$, as defined by equation (\ref{eq1})) (see Ref. \cite{ref1}).
\begin{comment}
\begin{equation}
\mathbb{X}^T \mathbb{X} \mathbb{B} = \mathbb{X}^T \mathbf{y}.\label{eq1}
\end{equation}
\end{comment}
\begin{equation}
\hat{\mathbf{x}} = \left(\mathbb{A}^T \mathbb{A}\right)^{-1} \mathbb{A}^T \mathbf{y}.\label{eq1}
\end{equation}


\section{General equation for a circle}
The general equation for a circle is
\begin{equation}
\left(x-x_0\right)^2 + \left(y-y_0\right)^2 = r^2,\label{eq2}
\end{equation}
where the center of the circle is defined at $\left(x_0, y_0\right)$, and $r$ is its radius.
Expansion of (\ref{eq2}) yields
\begin{equation}
x^2 -2x x_0 +x_0^2 + y^2 -2y y_0 +y_0^2 = r^2\label{eq3}.
\end{equation}
By rearranging (\ref{eq3}) so that
\begin{equation}
-2x x_0 -2y y_0 +x_0^2 +y_0^2 -r^2 = -\left(x^2 + y^2\right),\label{eq4}
\end{equation}
the generalized equation for a circle can be rearranged into the form of the matrix-vector equation (\ref{eq0}).

The variables to be solved ($x_0$, $y_0$, and $r$) must be packed into the vector $\mathbf{x}$.
Define
\begin{equation}
\mathbf{x} = 
\begin{bmatrix}
x_0 \\ y_0 \\ x_o^2 + y_0^2 -r^2
\end{bmatrix}_{ 3 \times 1},
\end{equation}
and rearrange equation (\ref{eq4}) so that  
\begin{equation}
\begin{bmatrix}
\begin{array}{c c c}
-2x &-2y  &+1
\end{array}
\end{bmatrix}
\begin{bmatrix}
x_0 \\ y_0 \\ x_o^2 + y_0^2 -r^2
\end{bmatrix} = 
\begin{bmatrix}
-\left(x^2 + y^2\right)
\end{bmatrix}.
\end{equation}
For a set of $n$ measurements, the remaining matrices to completely define (\ref{eq2}) in the form of (\ref{eq0}) are
\begin{equation}
\mathbb{A} = 
\begin{bmatrix}
\begin{array}{c c c}
-2x_1 &-2y_1 &+1 \\
-2x_2 &-2y_2 &+1 \\
-2x_3 &-2y_3 &+1 \\
\vdots	& \vdots & \vdots \\
-2x_n &-2y_n &+1 \\
\end{array}
\end{bmatrix}_{ n \times 3}
\end{equation}
and 
\begin{equation}
\mathbf{y} = 
\begin{bmatrix}
-\left(x_1^2+y_1^2\right)\\
-\left(x_2^2+y_2^2\right)\\
-\left(x_3^2+y_3^2\right)\\
\vdots\\
-\left(x_n^2+y_n^2\right)
\end{bmatrix}_{ n \times 1}
\end{equation}
For distinct values of $x_i$ and $y_i$, the columns of $\mathbb{A}$ are all linearly independent, satisfying the condition for (\ref{eq1}).
\section{General equation of a sphere}
The procedure is similar for a sphere. The general equation of a sphere is
\begin{equation}
\left(x-x_0\right)^2 + \left(y-y_0\right)^2 + \left(z-z_0\right)^2= r^2
\end{equation}
where the center of the sphere is defined at $\left(x_0, y_0, z_0\right)$ and $r$ is its radius.
Expansion yields
\begin{equation}
x^2 -2x x_0 +x_0^2 + y^2 -2y y_0 +y_0^2 + z^2 -2z z_0 +z_0^2= r^2,
\end{equation}
and rearranging yields
\begin{equation}
-2x x_0 -2y y_0 -2z z_0  +x_0^2 +y_0^2 +z_0^2 -r^2 = -\left(x^2 + y^2 + z^2\right),
\end{equation}
and the matrix-vector equation is now
\begin{equation}
\begin{bmatrix}
\begin{array}{c c c c}
-2x &-2y & -2z &+1
\end{array}
\end{bmatrix}
\begin{bmatrix}
x_0 \\ y_0 \\ z_0 \\ x_o^2 + y_0^2 +z_0^2 -r^2
\end{bmatrix} = 
\begin{bmatrix}
-\left(x^2 + y^2 +z^2\right)
\end{bmatrix}
\end{equation}
The matrices defining the least-squares solution are, therefore, 
\begin{equation}
\mathbb{A} = 
\begin{bmatrix}
\begin{array}{c c c c}
-2x_1 &-2y_1 &-2z_1 &+1 \\
-2x_2 &-2y_2 &-2z_2 &+1 \\
-2x_3 &-2y_3 &-2z_3 &+1 \\
\vdots	& \vdots & \vdots \\
-2x_n &-2y_n &-2z_n &+1 \\
\end{array}
\end{bmatrix}_{ n \times 4}
\end{equation}
\begin{equation}
\mathbf{x} = 
\begin{bmatrix}
x_0 \\ y_0 \\ z_0 \\ x_o^2 + y_0^2 + z_0^2 -r^2
\end{bmatrix}_{ 4 \times 1}
\end{equation}
and 
\begin{equation}
\mathbf{y}=
\begin{bmatrix}
-\left(x_1^2+y_1^2+z_1^2\right)\\
-\left(x_2^2+y_2^2+z_2^2\right)\\
-\left(x_3^2+y_3^2+z_3^2\right)\\
\vdots\\
-\left(x_n^2+y_n^2+z_n^2\right)
\end{bmatrix}_{ n \times 1}
\end{equation}
For distinct values of $x_i$, $y_i$, and $z_i$ the columns of $\mathbb{A}$ are all linearly independent, satisfying the condition for (\ref{eq1}).

\section{Weighted Least-squares}
Suppose each measurement is to be weighted. Define a weighting matrix as 
\begin{equation}
	\mathbb{W} = 
	\begin{bmatrix}
		\begin{array}{c c c c}
			w_1 &0  & \ldots &0 \\
			0 &w_2 & \ldots &0 \\
			\vdots &  & \ddots & \vdots \\
			0 & & \ldots & w_n \\
		\end{array}
	\end{bmatrix}_{n \times n}
\end{equation}
where each $w_i$ is applied to the $i^{th}$ measurement $y_i$.

The normal equation for the weighted-least squares solution is now (see Ref.\cite{ref2})
\begin{equation}
	\left(\mathbb{W}\mathbb{A}\right)^T \mathbb{W}\mathbb{A} \mathbf{x}= \left(\mathbb{W}\mathbb{A}\right)^T \mathbb{W} \mathbf{y}, 
\end{equation}
and if the columns of $\mathbb{W}\mathbb{A}$ are linearly independent, then there   exists only one least-squares solution for $\mathbf{x}$
\begin{thebibliography}{IEEEtran}
	\bibitem{ref1}
	D. C. Lay,								% author(s)
	\textit{Linear Algebra and Its Applications, 3rd Edition},				% title
	\textbf{Addison Wesley},				% publisher
	Chapter 6.5,								% volume, edition, and page number
	2003. 									% year
	\bibitem{ref2}
	D. C. Lay,								% author(s)
	\textit{Linear Algebra and Its Applications, 3rd Edition},				% title
	\textbf{Addison Wesley},				% publisher
	Chapter 6.8,								% volume, edition, and page number
	2003.
%	
%	\bibitem{ref2}
%	Ogata,									% author(s)
%	\textit{Modern Control Engineering},	% title
%	\textbf{Prentice Hall},					% publisher
%	2nd Edition, 							% volume, edition, and page number
%	1990									% year
\end{thebibliography}

\end{document}